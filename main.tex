% Copyright (c) 2008-2009 solvethis
% Copyright (c) 2010-2016 Casper Ti. Vector
% Public domain.
%
% 使用前请先仔细阅读 pkuthss 和 biblatex-caspervector 的文档,
% 特别是其中的 FAQ 部分和用红色强调的部分。
% 两者可在终端/命令提示符中用
%   texdoc pkuthss
%   texdoc biblatex-caspervector
% 调出。

% 采用了自定义的(包括大小写不同于原文件的)字体文件名,
% 并改动 ctex.cfg 等配置文件的用户请自行加入 nofonts 选项;
% 其它用户不用加入 nofonts 选项,加入之后反而会产生错误。
\documentclass[UTF8]{pkuthss}

% 使用 biblatex 排版参考文献,并规定其格式(详见 biblatex-caspervector 的文档)。
% 这里按照英文文献在前,中文文献在后排序(“sorting = ecnty”);
% 若需按照中文文献在前,英文文献在后排序,请设置“sorting = centy”;
% 若需按照引用顺序排序,请设置“sorting = none”。
% 若需在排序中实现更复杂的需求,请参考 biblatex-caspervector 的文档。
\usepackage[backend = biber, style = caspervector, utf8, sorting = none, maxbibnames=99]{biblatex}
\let\openbox\undefined
\usepackage{amsthm}
\usepackage{amsfonts}
\usepackage{amssymb,amsmath}
\usepackage[ruled,linesnumbered,algochapter]{algorithm2e}
\usepackage{subcaption}
\usepackage{tikz}
\usepackage{comment}
\usepackage{lscape}
\usepackage{listings,multicol} 
\usepackage{multirow}
\usepackage{enumitem}
\usepackage[perpage]{footmisc}

\usetikzlibrary{fit}
\makeatletter
\tikzset{
  fitting node/.style={
    inner sep=0pt,
    fill=none,
    draw=none,
    reset transform,
    fit={(\pgf@pathminx,\pgf@pathminy) (\pgf@pathmaxx,\pgf@pathmaxy)}
  },
  reset transform/.code={\pgftransformreset}
}
\makeatother

\definecolor{dkgreen}{rgb}{0,0.6,0}
\definecolor{gray}{rgb}{0.5,0.5,0.5}
\definecolor{mauve}{rgb}{0.58,0,0.82}
\definecolor{light-gray}{gray}{0.8}

\lstset{frame=tb,
  numbers=left,
  numbersep=2pt, %frame=line,
  language=Java,
  aboveskip=3mm,
  belowskip=3mm,
  showstringspaces=false,
  columns=flexible,
  basicstyle={\footnotesize\ttfamily},
  numberstyle=\scriptsize\color{gray},
  keywordstyle={\textbf},
  commentstyle=\color{dkgreen},
  stringstyle=\color{mauve},
  breaklines=true,
  breakatwhitespace=true,
  tabsize=4
}

% 按学校要求设定参考文献列表中的条目之内及之间的距离。
\setlength{\bibitemsep}{3bp}
% 对于 linespread 值的计算过程有兴趣的同学可以参考 pkuthss.cls。
\renewcommand*{\bibfont}{\zihao{5}\linespread{1.27}\selectfont}
\renewcommand{\algorithmcfname}{算法}

\newcommand{\CodeIn}[1]{{\texttt{#1}}}
\newcommand{\todo}[1]{{\color{red} \textbf{#1}}}

\newtheorem{definition}{\textbf{定义}}[chapter]
\newtheorem{example}{\textbf{例}}[chapter]
\newtheorem{theorem}{\textbf{定理}}[chapter]
\newtheorem{lemma}{\textbf{引理}}[chapter]

% 设定文档的基本信息。
\pkuthssinfo{
	cthesisname = {博士研究生学位论文}, ethesisname = {Doctor Thesis},
	ctitle = {面向某某问题的若干关键技术研究}, etitle = {Some Techniques for Some Problems},
	cauthor = {某 某},
	eauthor = {Bala Bala},
	studentid = {1901111666},
	date = {二〇一八年 十一月},
	school = {信息科学技术学院},
	cmajor = {计算机软件与理论}, emajor = {Computer Software and Theory},
	direction = {某某方向},
	cmentor = {某某某 教授}, ementor = {Prof.\ Bala Bala},
	ckeywords = {关键词一,关键词二}, ekeywords = {keyword 1, keyword 2}
}
% 载入参考文献数据库(注意不要省略“.bib”)。
\addbibresource{main.bib}

% 普通用户可删除此段,并相应地删除 chap/*.tex 中的
% “\pkuthssffaq % 中文测试文字。”一行。
\usepackage{color}
\def\pkuthssffaq{%
	\emph{\textcolor{red}{pkuthss 文档模版最常见问题:}}

	\texttt{\string\cite}、\texttt{\string\supercite} %
	和 \texttt{\string\supercite} 三个命令分别产生%
	未格式化的、带方括号的和上标且带方括号的引用标记:%
	\cite{test-en},\supercite{test-zh}、\supercite{test-en, test-zh}。

	若要避免章末空白页,请在调用 pkuthss 文档类时加入 \texttt{openany} 选项。

	如果编译时不出参考文献,
	请参考 \texttt{texdoc pkuthss}“问题及其解决”一章
	“其它可能存在的问题”一节中关于 biber 的说明。
}

\begin{document}
	% 以下为正文之前的部分,默认不进行章节编号。
	\frontmatter
	% 此后到下一 \pagestyle 命令之前不排版页眉或页脚。
	\pagestyle{empty}
	% 自动生成封面。
	\maketitle
	% 版权声明。封面要求单面打印,故需新开右页。
	\cleardoublepage
	% Copyright (c) 2008-2009 solvethis
% Copyright (c) 2010-2017 Casper Ti. Vector
% All rights reserved.
%
% Redistribution and use in source and binary forms, with or without
% modification, are permitted provided that the following conditions are
% met:
%
% * Redistributions of source code must retain the above copyright notice,
%   this list of conditions and the following disclaimer.
% * Redistributions in binary form must reproduce the above copyright
%   notice, this list of conditions and the following disclaimer in the
%   documentation and/or other materials provided with the distribution.
% * Neither the name of Peking University nor the names of its contributors
%   may be used to endorse or promote products derived from this software
%   without specific prior written permission.
%
% THIS SOFTWARE IS PROVIDED BY THE COPYRIGHT HOLDERS AND CONTRIBUTORS "AS
% IS" AND ANY EXPRESS OR IMPLIED WARRANTIES, INCLUDING, BUT NOT LIMITED TO,
% THE IMPLIED WARRANTIES OF MERCHANTABILITY AND FITNESS FOR A PARTICULAR
% PURPOSE ARE DISCLAIMED. IN NO EVENT SHALL THE COPYRIGHT HOLDER OR
% CONTRIBUTORS BE LIABLE FOR ANY DIRECT, INDIRECT, INCIDENTAL, SPECIAL,
% EXEMPLARY, OR CONSEQUENTIAL DAMAGES (INCLUDING, BUT NOT LIMITED TO,
% PROCUREMENT OF SUBSTITUTE GOODS OR SERVICES; LOSS OF USE, DATA, OR
% PROFITS; OR BUSINESS INTERRUPTION) HOWEVER CAUSED AND ON ANY THEORY OF
% LIABILITY, WHETHER IN CONTRACT, STRICT LIABILITY, OR TORT (INCLUDING
% NEGLIGENCE OR OTHERWISE) ARISING IN ANY WAY OUT OF THE USE OF THIS
% SOFTWARE, EVEN IF ADVISED OF THE POSSIBILITY OF SUCH DAMAGE.

% 此处不用 \specialchap,因为学校要求目录不包括其自己及其之前的内容。
\chapter*{版权声明}
% 综合学校的书面要求及 Word 模版来看,版权声明页不需加页眉、页脚。
\thispagestyle{empty}

任何收存和保管本论文各种版本的单位和个人,
未经本论文作者同意,不得将本论文转借他人,
亦不得随意复制、抄录、拍照或以任何方式传播。
否则一旦引起有碍作者著作权之问题,将可能承担法律责任。

% 若需排版二维码,请将二维码图片重命名为“barcode”,
% 转为合适的图片格式,并放在当前目录下,然后去掉下面 2 行的注释。
%\vfill\noindent
%\includegraphics[height = 5em]{barcode}

% vim:ts=4:sw=4


	% 此后到下一 \pagestyle 命令之前正常排版页眉和页脚。
	\cleardoublepage
	\pagestyle{plain}
	% 重置页码计数器,用大写罗马数字排版此部分页码。
	\setcounter{page}{0}
	\pagenumbering{Roman}
	% 中英文摘要。
	\include{chap/operators}
	% Copyright (c) 2014,2016 Casper Ti. Vector
% Public domain.

\begin{cabstract}
	传统静态分析工具往往存在\textbf{可扩展性问题},即难以高效地在大规模代码上执行静态分析。

\end{cabstract}

\begin{eabstract}
	Traditional static analysis tools suffer from \textbf{scalability} problem, i.e. cannot analyze large-scale code efficiently.

\end{eabstract}

	% 自动生成目录。
	\tableofcontents

	% 以下为正文部分,默认要进行章节编号。
	\mainmatter
	% 序言。
	% Copyright (c) 2014,2016 Casper Ti. Vector
% Public domain.

\chapter{引言}

静态分析是一项能够分析软件并保障代码质量的重要技术。这项技术从代码中获得软件行为,进而揭示缺陷代码和低效代码。开发人员可根据这些信息,对软件进行进一步的修复和优化。

\section{问题的提出}

\section{相关工作}

\section{本文主要工作}

\section{论文组织}

	% 各章节。
	\chapter{技术一}\label{chap:wildcard}

\section{引言}\label{sec:chap4intro}

\section{方法}

\section{实验}

\section{小结}

	% Copyright (c) 2014,2016 Casper Ti. Vector
% Public domain.

\chapter{技术二}\label{chapter:tal}

\section{算法示例}

算法~\ref{Algorithm:Chaining}中,如果一个结点$x$有一条标记为右括号的入边,另一个结点$y$有一条标记为左括号的出边,$x$到$y$存在一条路径,且这条路径上所有边的标记均为$e$,那么可以标记$y$为链结点。

\begin{algorithm} [ht]
\For {$x$是$G$的结点 $\wedge$ $x$至少有一条标记为右括号的入边} {
将$x$加入$C$; 将$x$加入$W$;
}
\While {$W$非空}{

取出并移除$W$的首个元素(记为$\varpi$);

\For {标记为$e$的$\varpi$的出边(记为$oe$)}{
将$oe$的标记变为右括号;\\
设$oe$将$\varpi$连结到$y$;\\
\If {$y\notin C$}{
将$y$加入$C$; 将$y$加入$W$;
}
}
}
\For {$y$是$G$的结点}{
\If {$y$至少有一条标记为右括号的入边 $\wedge$ $y$至少有一条标记为左括号的出边}{
将$y$标记为链结点;
}
}
\caption{识别链结点}
\label{Algorithm:Chaining}
\end{algorithm}

\section{图示例}

\textit{边界点.}
因为应用代码和摘要之间可达关系需要通过边界点进行传递,所以本文将所有\textit{边界点}作为关键摘要点。图\ref{fig:BoundaryNodes}\footnote{研究生院要求图标题放在图的后面。}展示了TALCRA摘要的四类边界点,其中的灰色结点表示摘要中的边界点,而白色表示应用代码结点。这四类结点的具体描述如下:
\begin{itemize}
\item 类型1边界点:如图\ref{fig:BoundaryNodes}(a)所示,应用代码通过库函数的调用将数值传给类型1边界点;
\item 类型2边界点:如图\ref{fig:BoundaryNodes}(b)所示,类型2边界点通过库函数的返回将数值传给应用代码;
\item 类型3边界点:如图\ref{fig:BoundaryNodes}(c)所示,类型3边界点通过回调函数的调用将数值传给应用代码;
\item 类型4边界点:如图\ref{fig:BoundaryNodes}(d)所示,应用代码通过回调函数的返回将数值传给类型4边界点。
\end{itemize}

\begin{figure}[htb]
\captionsetup[subfigure]{font=footnotesize}
\centering
\subcaptionbox{类型1边界点}[.25\textwidth]{%
	\begin{tikzpicture}[auto,->]
	\tikzstyle{cvertex}=[circle,draw=black,minimum size=10pt,inner sep=0pt]
	%\tikzstyle{cvertex}=[circle,draw=black,fill=white!15,minimum size=17pt,inner sep=0pt]
	\tikzstyle{lvertex}=[circle,fill=black!15,minimum size=10pt,inner sep=0pt]
	%\tikzstyle{lvertex}=[circle,fill=black!15,minimum size=17pt,inner sep=0pt]
	\tikzstyle{keyvertex}=[circle,thick,draw=black,fill=black!15,minimum size=10pt,inner sep=0pt]
	%\tikzstyle{keyvertex}=[circle,thick,draw=black,fill=black!15,minimum size=17pt,inner sep=0pt]
	\tikzstyle{weight} = [font=\small]
	\node[cvertex] (G-call) at (0,1.5) {};
	\node[keyvertex] (G-entry) at (1.5,1.5) {};
	\draw (G-call) -> node[weight] {$\{_i$} (G-entry);
	\end{tikzpicture}
}\label{fig:BoundaryNodes1}%
\subcaptionbox{类型2边界点}[.25\textwidth]{
	\begin{tikzpicture}[auto,->]
	\tikzstyle{cvertex}=[circle,draw=black,minimum size=10pt,inner sep=0pt]
	%\tikzstyle{cvertex}=[circle,draw=black,fill=white!15,minimum size=17pt,inner sep=0pt]
	\tikzstyle{lvertex}=[circle,fill=black!15,minimum size=10pt,inner sep=0pt]
	%\tikzstyle{lvertex}=[circle,fill=black!15,minimum size=17pt,inner sep=0pt]
	\tikzstyle{keyvertex}=[circle,thick,draw=black,fill=black!15,minimum size=10pt,inner sep=0pt]
	%\tikzstyle{keyvertex}=[circle,thick,draw=black,fill=black!15,minimum size=17pt,inner sep=0pt]
	\tikzstyle{weight} = [font=\small]
	\node[keyvertex] (G-exit) at (0,1.5) {};
	\node[cvertex] (G-return) at (1.5,1.5) {};
	\draw (G-exit) -> node[weight] {$\}_i$} (G-return);
	\end{tikzpicture}
}
\subcaptionbox{类型3边界点}[.25\textwidth]{%
	\begin{tikzpicture}[auto,->]
	\tikzstyle{cvertex}=[circle,draw=black,minimum size=10pt,inner sep=0pt]
	%\tikzstyle{cvertex}=[circle,draw=black,fill=white!15,minimum size=17pt,inner sep=0pt]
	\tikzstyle{lvertex}=[circle,fill=black!15,minimum size=10pt,inner sep=0pt]
	%\tikzstyle{lvertex}=[circle,fill=black!15,minimum size=17pt,inner sep=0pt]
	\tikzstyle{keyvertex}=[circle,thick,draw=black,fill=black!15,minimum size=10pt,inner sep=0pt]
	%\tikzstyle{keyvertex}=[circle,thick,draw=black,fill=black!15,minimum size=17pt,inner sep=0pt]
	\tikzstyle{weight} = [font=\small]
	\node[keyvertex] (G-call) at (0,1.5) {};
	\node[cvertex] (G-entry) at (1.5,1.5) {};
	\draw (G-call) -> node[weight] {$\{_i$} (G-entry);
	\end{tikzpicture}
}%
\subcaptionbox{类型4边界点}[.25\textwidth]{
	\begin{tikzpicture}[auto,->]
	\tikzstyle{cvertex}=[circle,draw=black,minimum size=10pt,inner sep=0pt]
	%\tikzstyle{cvertex}=[circle,draw=black,fill=white!15,minimum size=17pt,inner sep=0pt]
	\tikzstyle{lvertex}=[circle,fill=black!15,minimum size=10pt,inner sep=0pt]
	%\tikzstyle{lvertex}=[circle,fill=black!15,minimum size=17pt,inner sep=0pt]
	\tikzstyle{keyvertex}=[circle,thick,draw=black,fill=black!15,minimum size=10pt,inner sep=0pt]
	%\tikzstyle{keyvertex}=[circle,thick,draw=black,fill=black!15,minimum size=17pt,inner sep=0pt]
	\tikzstyle{weight} = [font=\small]
	\node[cvertex] (G-exit) at (0,1.5) {};
	\node[keyvertex] (G-return) at (1.5,1.5) {};
	\draw (G-exit) -> node[weight] {$\}_i$} (G-return);
	\end{tikzpicture}
}
\vspace{-2mm}
\caption{TALCRA摘要的四类边界点}\label{fig:BoundaryNodes}
%\vspace{-3mm}
\end{figure}

\section{表格示例}

表~\ref{tab:main}\footnote{研究生院要求表格标题放在表格前面。}展示了主要的实验结果。第2列是全程序的CFL分析实验效果。
第3到6列是应用代码分析的实验结果,第7到10列是构建库摘要的实验结果。

\begin{table}[h]
%\hspace{-10mm}
\center
\caption{\label{tab:main} 实验结果:TALCRA与CFLRA的运行时间和内存消耗对比}
\begin{tabular}{|c||r||rr|rr||rr|rr|}
\hline
程序&CFL&\multicolumn{4}{c||}{应用代码分析}&\multicolumn{4}{c|}{库代码摘要}\\\cline{3-10}
&&\multicolumn{2}{c|}{CFL}&\multicolumn{2}{c||}{TALCRA}&\multicolumn{2}{c|}{CFL}&\multicolumn{2}{c|}{TALCRA}\\
&总时间&时间&内存&时间&内存&时间&内存&时间&内存\\
&(ms)&(ms)&(MB)&(ms)&(MB)&(ms)&(MB)&(ms)&(MB)\\
\hline
\hline
check&615&348&152&67&58&267&107&645&132\\
compiler&639&330&153&53&73&309&130&661&141\\
compress&651&330&159&64&75&321&130&670&142\\
crypto&798&459&174&73&66&339&126&773&207\\
derby&1119&684&250&78&93&435&194&998&610\\
helloworld&592&309&130&36&48&283&105&640&138\\
mpegaudio&1622&1047&378&215&196&575&239&4421&387\\
scimark&653&331&161&67&78&322&131&665&149\\
startup&801&450&168&74&65&351&132&941&268\\
sunflow&583&290&123&33&44&293&100&663&127\\
xml&4378&2945&756&114&184&1433&587&5811&741\\
%\hline
%btree&105&80&48&56&179&68&264&155\\
%mushroom&80&64&8&48&169&76&183&93\\
%parser&108&86&13&53&199&74&269&158\\
%sample&61&64&13&48&162&73&184&99\\
\hline\hline
合计&12451&7523&&874&&4928&&16888&\\\hline
\end{tabular}
\end{table}

\section{代码示例}

图~\ref{fig:AbsClass}所示的上下文敏感数据依赖分析例子则表明,
库代码中的回调函数可能导致CFL可达性建立的库摘要产生漏报的情况。

\begin{figure}[htbp]
\begin{lstlisting}
package library;
public abstract class AbstractClass {
	public final int method1(int x1) {
		int y1 = x1 + 1;
		int z1 = method2(y1) + 1;
		return z1;
	}
	private final int method2(int x2) {
		int y2 = x2 + 2;
		int z2 = method3(y2) + 2;
		return z2;
	}
	abstract public int method3(int x3);
}
\end{lstlisting}
\caption{库代码示例}\label{fig:AbsClass}
\end{figure}

\section{引用示例}

\textit{延迟式摘要}是一个与\textit{条件式摘要}相对应的概念\footnote{研究生院要求每一页的脚注都要从①标起。}。
近期大多数摘要技术都属于\textit{延迟式摘要}技术,例如构件级别分析CLA(Component Level Analysis)~\supercite{rountev2006interprocedural,DBLP:conf/cc/RountevSX08}、数据结构分析~\supercite{DBLP:conf/pldi/LattnerLA07}、模块化堆分析~\supercite{Madhaven2012modular}和Android框架的StubDroid技术~\supercite{BoddenICSE16}\footnote{研究生院要求按照引用的顺序对引文进行编号,并且用上角标标注引文编号}。

%\pkuthssffaq 

	% 结论。
	% Copyright (c) 2014,2016 Casper Ti. Vector
% Public domain.

\chapter{结束语}\label{chapter:conclusion}
%\pkuthssffaq % 中文测试文字。
\section{本文工作总结}


	% 正文中的附录部分。
	\appendix
	% 排版参考文献列表。bibintoc 选项使“参考文献”出现在目录中;
	% 如果同时要使参考文献列表参与章节编号,可将“bibintoc”改为“bibnumbered”。
	\printbibliography[heading = bibintoc]
	% 各附录。

	% 以下为正文之后的部分,默认不进行章节编号。
	\backmatter
	% 致谢。
	\chapter{个人简历及博士期间研究成果}

\vspace{3pt}
\noindent{\large\textbf{个人简历}}
\vspace{6pt}


\vspace{20pt}
\noindent{\large\textbf{发表论文}}
\vspace{6pt}

\begin{enumerate}[label={[\arabic*]}]
	\item \textbf{A a}, B b,  and C c. “\textit{Six Idiots}”. In: \textit{Proceedings of the 66th Virtual Symposium , \textbf{VS 2018}, Shanghai, China, January 1-9, 2018}. 81–95.
\end{enumerate}

\vspace{20pt}
\noindent{\large\textbf{参与课题}}
\vspace{6pt}

\begin{enumerate}[label={[\arabic*]}]
	\item 千变万化,国家某基金项目,No. 12345678。
\end{enumerate}

	% Copyright (c) 2014,2016 Casper Ti. Vector
% Public domain.

\chapter{致谢}

	% 原创性声明和使用授权说明。
	% Copyright (c) 2008-2009 solvethis
% Copyright (c) 2010-2017 Casper Ti. Vector
% All rights reserved.
%
% Redistribution and use in source and binary forms, with or without
% modification, are permitted provided that the following conditions are
% met:
%
% * Redistributions of source code must retain the above copyright notice,
%   this list of conditions and the following disclaimer.
% * Redistributions in binary form must reproduce the above copyright
%   notice, this list of conditions and the following disclaimer in the
%   documentation and/or other materials provided with the distribution.
% * Neither the name of Peking University nor the names of its contributors
%   may be used to endorse or promote products derived from this software
%   without specific prior written permission.
%
% THIS SOFTWARE IS PROVIDED BY THE COPYRIGHT HOLDERS AND CONTRIBUTORS "AS
% IS" AND ANY EXPRESS OR IMPLIED WARRANTIES, INCLUDING, BUT NOT LIMITED TO,
% THE IMPLIED WARRANTIES OF MERCHANTABILITY AND FITNESS FOR A PARTICULAR
% PURPOSE ARE DISCLAIMED. IN NO EVENT SHALL THE COPYRIGHT HOLDER OR
% CONTRIBUTORS BE LIABLE FOR ANY DIRECT, INDIRECT, INCIDENTAL, SPECIAL,
% EXEMPLARY, OR CONSEQUENTIAL DAMAGES (INCLUDING, BUT NOT LIMITED TO,
% PROCUREMENT OF SUBSTITUTE GOODS OR SERVICES; LOSS OF USE, DATA, OR
% PROFITS; OR BUSINESS INTERRUPTION) HOWEVER CAUSED AND ON ANY THEORY OF
% LIABILITY, WHETHER IN CONTRACT, STRICT LIABILITY, OR TORT (INCLUDING
% NEGLIGENCE OR OTHERWISE) ARISING IN ANY WAY OUT OF THE USE OF THIS
% SOFTWARE, EVEN IF ADVISED OF THE POSSIBILITY OF SUCH DAMAGE.

{
	\ctexset{section = {
		format+ = {\centering}, beforeskip = {40bp}, afterskip = {15bp}
	}}

	% 学校书面要求本页面不要页码,但在给出的 Word 模版中又有页码且编入了目录。
	% 此处以 Word 模版为实际标准进行设定。
	\specialchap{北京大学学位论文原创性声明和使用授权说明}
	\mbox{}\vspace*{-3em}
	\section*{原创性声明}

	本人郑重声明:
	所呈交的学位论文,是本人在导师的指导下,独立进行研究工作所取得的成果。
	除文中已经注明引用的内容外,
	本论文不含任何其他个人或集体已经发表或撰写过的作品或成果。
	对本文的研究做出重要贡献的个人和集体,均已在文中以明确方式标明。
	本声明的法律结果由本人承担。
	\vskip 1em
	\rightline{%
		论文作者签名:\hspace{5em}%
		日期:\hspace{2em}年\hspace{2em}月\hspace{2em}日%
	}

	\section*{%
		学位论文使用授权说明\\[-0.33em]
		\textmd{\zihao{5}(必须装订在提交学校图书馆的印刷本)}%
	}

	本人完全了解北京大学关于收集、保存、使用学位论文的规定,即:
	\begin{itemize}
		\item 按照学校要求提交学位论文的印刷本和电子版本;
		\item 学校有权保存学位论文的印刷本和电子版,
			并提供目录检索与阅览服务,在校园网上提供服务;
		\item 学校可以采用影印、缩印、数字化或其它复制手段保存论文;
		\item 因某种特殊原因需要延迟发布学位论文电子版,
			授权学校在 $\Box$\nobreakspace{}一年 /
			$\Box$\nobreakspace{}两年 /
			$\Box$\nobreakspace{}三年以后在校园网上全文发布。
	\end{itemize}
	\centerline{(保密论文在解密后遵守此规定)}
	\vskip 1em
	\rightline{%
		论文作者签名:\hspace{5em}导师签名:\hspace{5em}%
		日期:\hspace{2em}年\hspace{2em}月\hspace{2em}日%
	}

	% 若需排版二维码,请将二维码图片重命名为“barcode”,
	% 转为合适的图片格式,并放在当前目录下,然后去掉下面 2 行的注释。
	%\vfill\noindent
	%\includegraphics[height = 5em]{barcode}
}

% vim:ts=4:sw=4

\end{document}

% vim:ts=4:sw=4
